\section{Ray Tracer}
\begin{Exercise}
Give the SDL file that produces this result:
\begin{center}
  \includegraphics[height=6cm]{scene.png}
\end{center}
The camera is located at $(0,0,-5)$, looks at $(0,0,0)$ and has up-vector $(0,1,0)$.
The sphere in the middle has radius 1; the lower sphere is twice as big.
Make sure not to forget about the following details:
\begin{itemize}
  \item The lights
  \item The material's properties (these do not have to be 100\% correct)
  \item The exact transformations.
\end{itemize}
\begin{solution}
\lstinputlisting[language={}]{scene.sdl}
\end{solution}
\end{Exercise}

\begin{Exercise}
Give the SDL file that produces this result:
\begin{center}
  \includegraphics[height=5cm]{scene2.png}
\end{center}
The camera is located at $(0,0,-5)$, looks at $(0,0,0)$ and has up-vector $(0,1,0)$.
The sphere in the middle has radius 1; the lower sphere is twice as big.
Make sure not to forget about the following details:
\begin{itemize}
  \item The lights
  \item The material's properties (these do not have to be 100\% correct)
  \item The exact transformations.
\end{itemize}
\begin{solution}
\lstinputlisting[language={}]{scene2.sdl}
\end{solution}
\end{Exercise}

\begin{Exercise}
Give the SDL file that produces this result:
\begin{center}
  \includegraphics[height=5cm]{scene3.png}
\end{center}
The camera is located at $(0,0,-5)$, looks at $(0,0,0)$ and has up-vector $(0,1,0)$.
The sphere in the middle has radius 1; the lower sphere is twice as big.
Make sure not to forget about the following details:
\begin{itemize}
  \item The lights
  \item The material's properties (these do not have to be 100\% correct)
  \item The exact transformations.
\end{itemize}
\begin{solution}
\lstinputlisting[language={}]{scene3.sdl}
\end{solution}
\end{Exercise}


%%% Local Variables:
%%% mode: latex
%%% TeX-master: "model-exam"
%%% End:
