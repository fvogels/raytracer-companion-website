\documentclass[oneside]{book}
\usepackage{amsthm}
\usepackage{a4wide}
\usepackage{booktabs}
\usepackage{microtype}
\usepackage{siunitx}
\usepackage{tikz}
\usepackage{pxfonts}
\usepackage{hyperref}
\usepackage{cleveref}
\usepackage{makeidx}
\usepackage{framed}
\usepackage{etoolbox}
\usepackage{comment}
\usepackage{ucll-uml}

\usetikzlibrary{arrows.meta,math,calc}

\newtoggle{showextra}
\ifdefined\examedition\togglefalse{showextra}\else\toggletrue{showextra}\fi

\newtheorem{definition}{Definition}[chapter]
\newtheorem{theorem}{Theorem}[chapter]
\newtheorem{exercise}{Exercise}[chapter]

\newenvironment{Definition}{\noindent\begin{samepage}\begin{definition}}{\end{definition}\end{samepage}}
\newenvironment{Exercise}{\noindent\begin{minipage}{.99\linewidth}\begin{exercise}\rm}{\end{exercise}\end{minipage}\vfill}

\newenvironment{example}{
  \vfill\begin{center}\begin{minipage}{.99\linewidth}\begin{framed}\begin{center}\begin{minipage}{.9\linewidth}\textsc{\bfseries Example}
}{
  \end{minipage}\end{center}\end{framed}\end{minipage}\end{center}\vfill
}

\newenvironment{solution}{
  \vskip1mm\noindent\textbf{Solution}
}{}

\iftoggle{showextra}{
  \includecomment{extra}
}{
  \excludecomment{extra}
}

\newcommand{\Z}{\mathbb{Z}}
\newcommand{\real}{\mathbb{R}}
\newcommand{\+}[1]{\mathrm{#1}}
\newcommand{\R}{\+{R}}
\newcommand{\G}{\+{G}}
\newcommand{\B}{\+{B}}
\newcommand{\norm}[1]{|#1|}
\newcommand{\conj}[1]{#1^*}
\newcommand{\transpose}[1]{#1^{\mathrm{T}}}
\newcommand{\inverse}[1]{#1^{-1}}
\newcommand{\degrees}{\ensuremath{^{\circ}}}
\newcommand{\colorrgb}[3]{\ensuremath{({\color{red}#1},{\color{green}#2},{\color{blue}#3})}}

\pgfkeys{
  /point/.cd,
  label/.initial={},
  position/.initial={(0,0)},
  anchor/.initial=south,
  dot radius/.initial=0.05,
}

\makeatletter

\newcommand{\point}[1][]{
  {
    \pgfkeys{
      #1,
      /point/label/.get=\point@label,
      /point/position/.get=\point@position,
      /point/anchor/.get=\point@anchor,
      /point/dot radius/.get=\point@radius
    }
    \draw[fill=black] \point@position circle (\point@radius) node[anchor=\point@anchor] {\point@label};
  }
}

\makeatother

\tikzset{
  vector/.style={-latex,thick},
  light/.style={-latex,thick,orange},
  axis/.style={-latex,thick}
}

\makeindex

\begin{document}

\begin{extra}
\tableofcontents
\end{extra}

\chapter{Trigonometry}
\begin{Definition}[sine and cosine]
The \emph{cosine} and \emph{sine} of an angle $\theta$ are the
$x$ and $y$ coordinate of the following point on a unit circle:
\begin{center}
  \begin{tikzpicture}
    \draw[-latex,thick] (-3,0) -- (3,0);
    \draw[-latex,thick] (0,-3) -- (0,3);
    \draw (0,0) circle (2);

    \draw (0,0) -- (60:3);
    \draw (0:.5) arc [start angle=0,end angle=60,radius=0.5] node[midway,anchor=south west,font=\tiny,inner sep=1mm] {$\theta$};

    \draw[dotted] let \p1=(60:2) in (\p1) -- (\x1,0) node[anchor=north,font=\tiny] {$\cos \theta$};
    \draw[dotted] let \p1=(60:2) in (\p1) -- (0,\y1) node[anchor=east,font=\tiny] {$\sin \theta$};
  \end{tikzpicture}
\end{center}
\end{Definition}

Drawing both these trigonometric functions yields the following graph:
\begin{center}
  \begin{tikzpicture}
    \draw[-latex,thick] (-5,0) -- (5,0);
    \draw[-latex,thick] (0,-1.5) -- (0,1.5);

    \draw[blue,domain=-5:5,smooth,samples=100] plot (\x,{sin(180 * \x)});
    \draw[red,domain=-5:5,smooth,samples=100] plot (\x,{cos(180 * \x)});

    \foreach \x in {-4,-2,-1,1,2,4} {
      \tikzmath{
        real \c;
        \c = int(\x * 180);
      }
      \draw (\x,-0.1) -- (\x,0.1);
      \node[anchor=north,font=\tiny] at (\x,0) {\c\degrees};
    }

    \draw (-0.05,1) -- (0.05,1);
    \draw (-0.05,-1) -- (0.05,-1);
    \node[anchor=west,font=\tiny] at (0,1) {1};
    \node[anchor=west,font=\tiny] at (0,-1) {-1};

    \node (cos) at (-5,2) {\color{red}$\cos$};
    \node[anchor=base west] (sin) at (cos.base east) {\color{blue}$\sin$};
  \end{tikzpicture}
\end{center}

Some noteworthy values are
\[
  \begin{array}{ccc}
    \theta & \cos\theta & \sin\theta \\
    \toprule
    0\degrees & 1 & 0 \\
    90\degrees & 0 & 1 \\
    180\degrees & -1 & 0 \\
    270\degrees & 0 & -1 \\
  \end{array}
\]

\begin{theorem} \label{thm:sin-cos-relation}
For any angle $\theta$,
\[
  \cos(\theta)^2 + \sin(\theta)^2 = 1
\]
\end{theorem}

%%% Local Variables:
%%% mode: latex
%%% TeX-master: "reference"
%%% End:

\chapter{3D Geometry}

\begin{Definition}[point] \index{point}
A \emph{point} $P(x, y, z) \in \real$ describes a position in space.
\end{Definition}

\begin{Definition}[vector] \index{vector}
A \emph{vector} $\vec v(x, y, z) \in \real$ describes a displacement in space.
\end{Definition}

\begin{Definition}[addition] \index{addition!vector}\index{vector!addition}
The addition of two points/vectors is equal to
\[
  (x,y,z) + (x',y',z') = (x+x',y+y',z+z')
\]
Addition is possible on the following operand types:
\begin{center}
  \begin{tabular}{ccc}
    \textbf{left operand} & \textbf{right operand} & \textbf{result} \\
    \toprule
    point & vector & point \\
    vector & point & point \\
    vector & vector & vector \\
  \end{tabular}
\end{center}
\end{Definition}

\begin{extra}
  \begin{example}
  Adding a point $P(1,1,0)$ with a vector $\vec v(2,1,0)$ corresponds to moving
  $P$ in a direction and over a distance specified by $\vec v$.
  \begin{center}
    \begin{tikzpicture}
      \draw[thin,gray] (0,0) grid (4,4);
      \coordinate (p) at (1,1);
      \coordinate (q) at (3,2);

      \draw[-latex,thick] (0,0) -- (4,0);
      \draw[-latex,thick] (0,0) -- (0,4);

      \point[/point/label={$P$},/point/position={(p)}]
      \draw[vector] (p) -- (q) node[midway,sloped,above] {$\vec v$};
      \point[/point/label={$P+\vec v$},/point/position=(q)]
    \end{tikzpicture}
  \end{center}
  \end{example}
\end{extra}

\begin{extra}
  \begin{example}
  Adding two vectors $\vec u$ and $\vec v$ corresponds to combining
  both displacements into one.
  \begin{center}
    \begin{tikzpicture}
      \draw[thin,gray] (0,0) grid (4,4);

      \draw[-latex,thick] (0,0) -- (4,0);
      \draw[-latex,thick] (0,0) -- (0,4);

      \draw[vector] (1,1) -- (2,1) node[midway,above] {$\vec u$};
      \draw[vector] (2,1) -- (2,3) node[midway,right] {$\vec v$};
      \draw[vector] (1,1) -- (2,3) node[midway,sloped,above] {$\vec u+\vec v$};
    \end{tikzpicture}
  \end{center}
  \end{example}
\end{extra}

\begin{Definition}[subtraction] \index{subtraction!vector}\index{vector!subtraction}
The subtraction of two points/vectors is equal to
\[
  (x,y,z) - (x',y',z') = (x-x',y-y',z-z')
\]
Subtraction is possible on the following operand types:
\begin{center}
  \begin{tabular}{ccc}
    \textbf{left operand} & \textbf{right operand} & \textbf{result} \\
    \toprule
    point & point & vector \\
    point & vector & point \\
    vector & vector & vector \\
  \end{tabular}
\end{center}
\end{Definition}

\begin{Definition}[scalar multiplication] \index{multiplication}\index{scalar multiplication}
The scalar multiplication of a scalar $f \in \real$ and a vector $\vec v(x,y,z)$ is equal to
\[
  f \cdot (x, y, z) = (x, y, z) \cdot f = (f \cdot x, f \cdot y, f \cdot z)
\]
\end{Definition}

\begin{extra}
  \begin{example}
  Scalar multiplication corresponds to stretching or shrinking the vector: its orientation
  remains the change but its length is modified.
  \begin{center}
    \begin{tikzpicture}
      \draw[vector] (0,0) -- (20:2) node[anchor=south] { $\vec v$ };
      \draw[vector,yshift=-1mm] (0,0) -- (20:4) node[anchor=south] { $2 \vec v$ };
      \draw[vector,yshift=-2mm] (0,0) -- (20:6) node[anchor=south] { $3 \vec v$ };
      \draw[vector,yshift=-3mm] (0,0) -- (20:-2) node[anchor=south] { $-\vec v$ };
      \draw[vector,yshift=-4mm] (0,0) -- (20:-4) node[anchor=south] { $-2\vec v$ };
      \draw[vector,yshift=-5mm] (0,0) -- (20:-6) node[anchor=south] { $-3\vec v$ };
    \end{tikzpicture}
  \end{center}
  \end{example}
\end{extra}

\begin{Definition}[length of a vector] \index{vector!length}\index{length}
The \emph{length} of a vector $\vec v(x, y, z)$ equals
\[
  \norm{\vec{v}} = \sqrt{x^2+y^2+z^2}
\]
\end{Definition}

\begin{Definition}[dot product] \label{def:dot-product} \index{vector!dot product}\index{dot product}\index{multiplication!dot product}
The \emph{dot product} of two vectors $\vec{u}(x, y, z)$ and $\vec{v}(x', y', z')$ is equal to
\[
  \vec u \cdot \vec v = x \cdot x' + y \cdot y' + z \cdot z' = \norm{\vec u} \cdot \norm{\vec v} \cdot \cos \theta
\]
where $\theta$ is the angle between the two vectors.
\begin{center}
  \begin{tikzpicture}
    \draw[vector] (0,0) -- (30:2);
    \draw[vector] (0,0) -- (60:3);
    \draw (30:1) arc [start angle=30,end angle=60,radius=1cm] node[midway,anchor=south west] {$\theta$};
  \end{tikzpicture}
\end{center}
\end{Definition}

\begin{extra}
  \begin{example}
  We can use the dot product to determine the angle between vectors:
  \[
    \vec{u} \cdot \vec{v} = \norm{\vec u} \cdot \norm{\vec v} \cdot \cos\theta
    \qquad\Rightarrow\qquad
    \cos\theta = \frac{\vec u \cdot \vec v}{\norm{\vec u} \cdot \norm{\vec v}}
  \]
  We apply this on the following vectors:
  \begin{center}
    \begin{tikzpicture}
      \draw[vector] (0,0) -- (2,1) node[anchor=south] {$\vec u = (2,1,0)$};
      \draw[vector] (0,0) -- (1,2) node[anchor=south] {$\vec v = (1,2,0)$};
    \end{tikzpicture}
  \end{center}
  \[
    \cos\theta = \frac{\vec u \cdot \vec v}{\norm{\vec u} \cdot \norm{\vec v}}
               = \frac{2 + 2 + 0}{\sqrt{5} \sqrt{5}}
               = \frac{4}{5}
    \qquad\Rightarrow\qquad
    \theta = \arccos \frac45 = 36.87\degrees
  \]
  \end{example}
\end{extra}

\begin{theorem}
Two nonzero length vectors $\vec u$ and $\vec u$ are perpendicular iff $\vec u \cdot \vec v = 0$.
\end{theorem}
\begin{extra}
  \begin{proof}
  According to \cref{def:dot-product}
  \[
    \vec u \cdot \vec v = \norm{\vec u} \cdot \norm{\vec v} \cdot \cos \theta
  \]
  A product is zero if one of its factors is zero. Given that both vectors have nonzero lengths ($\norm{\vec u} \neq = 0$ and $\norm{\vec v} \neq 0$),
  the only remaining possibility is that $\cos \theta = 0$. This is the case when $\theta = 90\degrees + k \cdot 180\degrees$ with $k \in \Z$, all
  of which correspond to perpendicularity.
  \end{proof}
\end{extra}

\begin{theorem}
For any vector $\vec v$,
\[
  \norm{\vec v}^2 = \vec v \cdot \vec v
\]
\end{theorem}
\begin{extra}
  \begin{proof}
  From \cref{def:dot-product},
  \[
    \vec v \cdot \vec v = \norm{\vec v} \cdot \norm{\vec v} \cdot \cos \theta
  \]
  A vector always makes an angle of $0 \degrees$ with itself. Since $\cos 0\degrees = 1$, we get
  \[
    \vec v \cdot \vec v = \norm{\vec v} \cdot \norm{\vec v} = \norm{\vec v}^2
  \]
  \end{proof}
\end{extra}

\begin{theorem} \label{thm:dot-product-projection}
The dot product $\vec u \cdot \vec v$ is equal to $\norm{\vec u} \cdot \norm{\vec v_\+p}$,
where $\vec v_\+p$ is the projection of $\vec v$ on $\vec u$.
\begin{center}
  \begin{tikzpicture}[rotate=20]
    \draw[dotted] let \p1=(30:4) in (\p1) -- (\x1,0) -- (0,0);
    \draw[vector,yshift=-2mm] let \p1=(30:4) in (0,0) -- (\x1,0) node[anchor=north] {$\vec v_\+p$};
    \draw[vector] (0,0) -- (2,0) node[anchor=south] {$\vec u$};
    \draw[vector] (0,0) -- (30:4) node[anchor=south] {$\vec v$};
  \end{tikzpicture}
\end{center}
\end{theorem}

\begin{Definition}[cross product] \index{cross product}\index{vector!cross product}\index{multiplication!cross product}
The \emph{cross product} of two vectors $\vec u(x, y, z)$ and $\vec{v}(x', y', z')$ is equal to
\[
  \vec u \times \vec v =
  \left|
    \begin{array}{ccc}
      \vec{X} & \vec{Y} & \vec{Z} \\
      x & y & z \\
      x' & y' & z'
    \end{array}
  \right|
  =
 \left( \begin{array}{r@{\;}c@{\;}l}
          y z' - y' z \\
          x' z - x z' \\
          xy' - x'y \\
        \end{array} \right)
\]
\end{Definition}

\begin{theorem}
Given two vectors $\vec{u}(x, y, z)$ and $\vec{v}(x', y', z')$, their cross product $\vec u \times \vec v$ is perpendicular on both of them.
\end{theorem}
\begin{extra}
  \begin{proof}
  We need to prove that
  \[
    (\vec u \times \vec v) \cdot \vec u = 0 \qquad (\vec u \times \vec v) \cdot \vec v = 0
  \]
  We prove the left equation:
  \[
    \begin{array}{rcl}
      (y z' - y' z, x' z - x z', xy' - x'y) \cdot (x, y, z) & = & x \cdot (y z'-y' z) + y \cdot (x' z-x z') + z \cdot (xy' - x'y) \\ 
                                                            & = & x y z' - x y' z + x' y z - x y z' + x y' z - x' y z \\
                                                            & = & 0
    \end{array}
  \]
  Idem for the right equation.
  \end{proof}
\end{extra}

\begin{Definition}[ray] \index{ray}
A \emph{ray} consists of all points on a straight line.
A ray is uniquely defined by a starting point $P$ and a (nonzero) direction vector $\vec d$.
The set of all points can then be written as
\[
  \{ \; P + t \cdot \vec d \;|\; t \in \real \; \}
\]
\begin{center}
  \begin{tikzpicture}
    \draw[-latex,thick] (0,0) -- (5,0);
    \draw[-latex,thick] (0,0) -- (0,4);

    \point[/point/label=P,/point/position={(1,1)}]
    \draw[vector] (1,1) -- +(30:1) node[midway,sloped,above] {$\vec d$};
    \draw (1,1) -- +(210:2.5);
    \draw (1,1) -- ++(30:4.5);

    \coordinate (q) at ($ (1,1) + (30:1) $);
    \foreach \t in {-2,-1,0,1,2,3,4} {
      \coordinate (p) at ($ (1,1) ! \t ! (q) $);
      \draw[fill=black] (p) circle (.025);
      \node[font=\tiny,anchor=north west,inner sep=1pt] at (p) {t=\t};
    }
  \end{tikzpicture}
\end{center}
\end{Definition}

\begin{Definition}[plane] \index{plane}
A \emph{plane} is unambiguously defined by a point $P$ and a (nonzero) normal vector $\vec n$.
The set of all points on a plane can be written
\[
  \{ \; Q \;|\; (Q-P) \cdot \vec n = 0 \; \}
\]
i.e.\ the set of all points for which the vector $\overrightarrow{PQ} = Q-P$ is perpendicular on $\vec n$.
\end{Definition}

\begin{theorem}[ray-plane intersection] \label{thm:ray-plane}
The intersection of a ray with starting point $P$ and direction vector $\vec d$
and a plane with point $Q$ and normal vector $\vec n$ can be found using the following formula:
\[
  P + \frac{(Q-P) \cdot \vec n}{\vec d \cdot \vec n} \cdot \vec d
\]
If $\vec d \cdot \vec n = 0$, there is no intersection.
\end{theorem}
\begin{extra}
  \begin{proof}
  In order to find the intersection, we need to solve the following equation for $t$:
  \[
    ((P + t \cdot \vec d) - Q) \cdot \vec n = 0
  \]
  Expanding gives us
  \[
    (P - Q) \cdot \vec n + t \cdot (\vec d \cdot \vec n) = 0
  \]
  Solving for $t$
  \[
    t = \frac{(Q-P) \cdot \vec n}{\vec d \cdot \vec n}
  \]
  Plugging in $t$ in $P + t \cdot \vec d$ yields the desired result.
  \end{proof}
\end{extra}

\begin{extra}
  \begin{example}
  We consider a situation where we can easily determine the intersection ourselves.
  Take the XY plane, i.e.\ point $Q = (0,0,0)$ and $\vec n = (0,0,1)$.
  Take a ray parallel to the Z-axis shooting straight down towards the plane, i.e.\ starting point $P = (5,3,7)$
  and $\vec d = (0,0,-1)$. It should hit the plane at $(5,3,0)$.

  Applying the formula from \cref{thm:ray-plane} gives us
  \[
    \begin{array}{rl}
      & \displaystyle (5,3,7) + \frac{((0,0,0) - (5,3,7)) \cdot (0,0,1)}{(0,0,-1)\cdot(0,0,1)} \cdot (0,0,-1) \\ \\
      = & \displaystyle (5,3,7) + \frac{(-5,-3,-7) \cdot (0,0,1)}{(0,0,-1)\cdot(0,0,1)} \cdot (0,0,-1) \\ \\
      = & \displaystyle (5,3,7) + \frac{-7}{-1} \cdot (0,0,-1) \\ \\
      = & \displaystyle (5,3,7) + (0,0,-7) \\ \\
      = & \displaystyle (5,3,0)
    \end{array}
  \]
  \end{example}
\end{extra}

\begin{extra}
  \begin{example}
  We can generalise the previous example in the following ways:
  \begin{itemize}
    \item It does not matter which point $Q$ of the plane we take. Any point $(\alpha, \beta, 0)$ with $\alpha,\beta \in \real$ will do.
    \item The normal vector $\vec n$ can be of any length, so we generalise it to $\vec n = (0,0,\gamma)$ with $\gamma \in \real_0$.
    \item The height of the ray's starting position $P$ does not matter: we can take $P = (5,3,\delta)$ with $\delta \in \real$.
    \item The length of the direction vector does not matter, so we take $d = (0,0,\epsilon)$ with $\epsilon \in \real_0$.
  \end{itemize}
  Applying the formula from \cref{thm:ray-plane} should still give us the exact same result:
  \[
    \begin{array}{rl}
      & \displaystyle (5,3,\delta) + \frac{((\alpha,\beta,0) - (5,3,\delta)) \cdot (0,0,\gamma)}{(0,0,\epsilon)\cdot(0,0,\gamma)} \cdot (0,0,\epsilon) \\ \\
      = & \displaystyle (5,3,\delta) + \frac{-\delta \cdot \gamma}{\epsilon \cdot \gamma} \cdot (0,0,\epsilon) \\ \\
      = & \displaystyle (5,3,\delta) + \frac{-\delta}{\epsilon} \cdot (0,0,\epsilon) \\ \\
      = & \displaystyle (5,3,\delta) + (0,0,-\delta) \\ \\
      = & \displaystyle (5,3,0)
    \end{array}
  \]
  \end{example}
\end{extra}

\begin{Definition}[sphere] \index{sphere}
A \emph{sphere} is the set of all points at an equal distance from a centre point.
A sphere is uniquely defined by its center $C$ and its radius $r$. The set
of all points on a sphere can be written
\[
  \{ \; P \;|\; (P-C) \cdot (P-C) = r^2 \; \}
\]
\end{Definition}

\begin{theorem}[ray-sphere intersection]
Given a ray with starting position $P$ and direction vector $\vec d$.
Given a sphere with centre $O(0,0,0)$ and radius 1.
To find the intersections between these two, perform the following steps:
\begin{enumerate}
  \item Compute $a = \vec d \cdot \vec d$
  \item Compute $b = 2 \cdot (P - O) \cdot \vec d$
  \item Compute $c = (P - O) \cdot (P - O) - r ^ 2$
  \item Compute $D = b^2 - 4 \cdot a \cdot c$
  \item If $D < 0$, there are no intersections.
  \item If $D = 0$, the intersection is located at
        \[
          P + \frac{-b}{2 \cdot a} \cdot \vec d
        \]
  \item If $D > 0$, there are two intersections at
        \[
          P + \frac{-b \pm \sqrt D}{2 \cdot a} \cdot \vec d
        \]
\end{enumerate}
\end{theorem}
\begin{extra}
  \begin{proof}
  All points on the rays can be written $P + t \cdot \vec d$.
  All points $Q$ on the sphere must satisfy $(Q - O) \cdot (Q - O) = 1$.
  Putting these two things together gives
  \[
    (P + t \cdot \vec d - O) \cdot (P + t \cdot \vec d - O) = 1
  \]
  Expanding
  \[
    (P - O) \cdot (P - O) + \vec d \cdot \vec d \cdot t^2 + 2 \cdot (P - O) \cdot \vec d \cdot t = 1
  \]
  We rearrange the terms
  \[
    (\vec d \cdot \vec d) \cdot t^2 + (2 \cdot (P - O) \cdot \vec d) \cdot t + (P - O) \cdot (P - O) - 1 = 0
  \]
  We perform the following substitutions:
  \[
    \begin{array}{rcl}
      a & = & \vec d \cdot \vec d \\
      b & = & 2 \cdot (P - O) \cdot \vec d \\
      c & = & (P - O) \cdot (P - O) - 1 \\
    \end{array}
  \]
  yielding
  \[
    a \cdot t^2 + b \cdot t + c = 0
  \]
  This is a quadratic equation. The discriminant is
  \[
    D = b^2 - 4 \cdot a \cdot c
  \]
  \begin{itemize}
    \item If $D < 0$, the ray misses the sphere and there are no intersections.
    \item If $D = 0$, the ray grazes the sphere and there is one intersection at $t = \frac{-b}{2 \cdot a}$.
    \item If $D > 0$, the ray enters and exists the sphere in two different intersection points at
          \[
            t = \frac{-b \pm \sqrt D}{2 \cdot a}
          \]
  \end{itemize}
  \end{proof}
\end{extra}

\begin{extra}
  \begin{example}
  Consider the ray with starting point $P = (0,0,5)$ and direction vector $\vec d = (0,0,-1)$. We expect
  the ray to hit the sphere in places, namely $(0,0,1)$ and $(0,0,-1)$.
  We perform the necessary computations:
  \begin{itemize}
    \item $a = \vec d \cdot \vec d = 1$.
    \item $b = 2 \cdot (P - O) \cdot \vec d = 2 \cdot (0,0,5) \cdot (0,0,-1) = -10$.
    \item $c = (P-O) \cdot (P-O) - r^2 = (0,0,5) \cdot (0,0,5) - 1^2 = 24$.
    \item $D = b^2 - 4 \cdot a \cdot c = 100 - 96 = 4$.
  \end{itemize}
  Since $D > 0$, there are two intersections $I_1$ and $I_2$.
  \[
    \begin{array}{r@{\;}c@{\;}l}
      I_1 & = & \displaystyle P + \frac{-b-\sqrt D}{2 \cdot a} \cdot \vec d = (0,0,5) + \frac{10 - 2}{2} \cdot (0,0,-1) = (0,0,5) + (0,0,-4) = (0,0,1) \\ \\
      I_2 & = & \displaystyle P + \frac{-b+\sqrt D}{2 \cdot a} \cdot \vec d = (0,0,5) + \frac{10 + 2}{2} \cdot (0,0,-1) = (0,0,5) + (0,0,-6) = (0,0,-1) \\ \\
    \end{array}
  \]
  \end{example}
\end{extra}

\begin{extra}
  \begin{exercise}
  Find the intersections of a ray and a cylinder.
  \end{exercise}
\end{extra}

\begin{extra}
  \begin{exercise}
  Find the intersections of a ray and a cone.
  \end{exercise}
\end{extra}

\begin{theorem}[reflection] \label{thm:reflection} \index{reflection}
The \emph{reflection} of a vector $\vec v$ by a unit vector $\vec n$ can be computed as follows:
\[
  \vec v - 2 \cdot (\vec v \cdot \vec n) \cdot \vec n
\]
\end{theorem}
\begin{extra}
  \begin{proof}
  We make a quick sketch of the situation:
  \begin{center}
    \begin{tikzpicture}
      \draw[ultra thick] (-2,0) -- (2,0);
      \draw[vector] (0,0) -- (0,1) node[anchor=south] { $\vec n$ };
      \draw[vector] (120:2) -- (0,0) node[at start,anchor=south] { $\vec v$ };
      \draw[vector] (0,0) -- (60:2) node[at end,anchor=south] { $\vec r$ };
    \end{tikzpicture}
  \end{center}
  We decompose $v$ in a horizontal and vertical component:
  \begin{center}
    \begin{tikzpicture}
      \draw[ultra thick] (-2,0) -- (2,0);
      \draw[vector] (0,0) -- (0,1) node[anchor=south] { $\vec n$ };
      \draw[vector] (120:2) -- (0,0) node[at start,anchor=south] { $\vec v$ };
      \draw[vector] let \p1=(120:2) in (\p1) -- (\x1,0) node[midway,anchor=east] { $\vec v_\+y$ };
      \draw[vector] let \p1=(120:2) in (\p1) -- (0,\y1) node[midway,anchor=south] { $\vec v_\+x$ };
      \draw[vector] (0,0) -- (60:2) node[at end,anchor=south] { $\vec r$ };
    \end{tikzpicture}
  \end{center}
  We can see that $\vec r = \vec v_\+x - \vec v_\+y$. If we can find a way to determine
  these $\vec v_\+x$ and $\vec v_\+y$, we're set. We cannot just take the X- and Y-coordinate,
  as it is possible that the reflecting surface is not perfectly horizontal:
  \begin{center}
    \begin{tikzpicture}[rotate=30]
      \draw[ultra thick] (-2,0) -- (2,0);
      \draw[vector] (0,0) -- (0,1) node[anchor=south] { $\vec n$ };
      \draw[vector] (120:2) -- (0,0) node[at start,anchor=south] { $\vec v$ };
      \draw[vector] (0,0) -- (60:2) node[at end,anchor=south] { $\vec r$ };
    \end{tikzpicture}
  \end{center}
  We can find $\vec v_\+y$ by projecting $\vec v$ onto $\vec n$ (see \cref{thm:dot-product-projection}):
  \[
    \vec v_\+y = (\vec v \cdot \vec n) \cdot \vec n
  \]
  We can then easily find $\vec v_\+x$ as follows:
  \[
    \vec v_\+x = \vec v - \vec v_\+y
  \]
  Substituting yields us the final result:
  \[
    \vec r = \vec v_\+x - \vec v_\+y
           = \vec v - \vec v_\+y - \vec v_\+y
           = \vec v - 2 \cdot \vec v_\+y
           = \vec v - 2 \cdot (\vec v \cdot \vec n) \cdot \vec n
  \]
  \end{proof}
\end{extra}

\begin{extra}
  \begin{example}
  \begin{center}
    \begin{tikzpicture}
      \draw[vector] (0,0) -- (1,0) node[at end,anchor=west] { $\vec n = (1,0,0)$ };
      \draw[vector] (2,1) -- (0,0) node[at start,anchor=south] { $\vec v = (-2,-1,0)$ };
      \draw[vector] (0,0) -- (2,-1) node[at end,anchor=north] { $\vec r$ };
    \end{tikzpicture}
  \end{center}
  We compute $\vec r$'s coordinates using \cref{thm:reflection}.
  \[
    \begin{array}{r@{\;}c@{\;}l}
      \vec v - 2 \cdot (\vec v \cdot \vec n) \cdot \vec n
      & = & (-2,-1,0) - 2 \cdot ((-2,-1,0) \cdot (1,0,0)) \cdot (1,0,0) \\
      & = & (-2,-1,0) - 2 \cdot -2 \cdot (1,0,0) \\
      & = & (-2,-1,0) + (4,0,0) \\
      & = & (2,-1,0)
    \end{array}
  \]
  \end{example}
\end{extra}

%%% Local Variables:
%%% mode: latex
%%% TeX-master: "reference"
%%% End:

\chapter{Complex Numbers and Quaternions}
\begin{Definition}[imaginary unit] \index{imaginary number}\index{$i$}
The \emph{imaginary unit} $i$ is a value satisfying $i^2 = -1$.
\end{Definition}

\begin{Definition}[complex number] \index{complex number}
A \emph{complex number} is a number of the form $a + b i$ with $a, b \in \real$.
\end{Definition}

\begin{Definition}[quaternion] \index{conjugate}
A \emph{quaternion} is a number of the form $a + bi + cj + dk$ with $a,b,c,d \in \real$ and $i,j,k$ satisfying the following properties
\[
  \begin{array}{c|ccc}
    \times & i & j & k \\
    \hline
    i & -1 & k & -j \\
    j & -k & -1 & i \\
    k & j & -i & -1
  \end{array}
\]
Note that these multiplications are \emph{not} commutative.
\end{Definition}


\begin{Definition}[conjugate] \index{conjugate}\index{quaternion!conjugate}
The \emph{conjugate} of a quaternion $q = a+bi+cj+dk$ is defined as
\[
  \conj{q} = a -bi-cj-dk
\]
\end{Definition}

\begin{Definition}[rotation quaternion] \index{rotation quaternion}\index{quaternion!rotation}
The \emph{rotation quaternion} for a rotation of angle $\theta$ around
an axis represented by the unit vector $(r_x, r_y, r_z)$ is defined as
\[
  q = \cos\left(\frac\theta2\right) +
      \sin\left(\frac\theta2\right) r_x i + 
      \sin\left(\frac\theta2\right) r_y j + 
      \sin\left(\frac\theta2\right) r_z k
\]
\end{Definition}

\begin{theorem} \index{rotation}
In order to apply a rotation represented by a quaternion $q$ on a point $P(x,y,z)$, perform
the following steps:
\begin{enumerate}
  \item Let $p = xi+yj+zk$.
  \item Let $p' = q \cdot p \cdot \conj{q}$ which is of the form $x'i+y'j+z'k$.
  \item The result of the rotation is $(x',y',z')$.
\end{enumerate}
\end{theorem}

\begin{extra}
  \begin{example}
  Rotating $P(2,0,0)$ $180\degrees$ around the axis $(0,1,0)$ should yield $(-2,0,0)$.
  The rotation quaternion equals
  \[
    q = \cos\left(90\degrees\right) +
        \sin\left(90\degrees\right) \cdot 0 \cdot i + 
        \sin\left(90\degrees\right) \cdot 1 \cdot j + 
        \sin\left(90\degrees\right) \cdot 0 \cdot k = j
  \]
  We have $p = 2i+0j+0k = 2i$. We perform quaternion multiplication:
  \[
    q \cdot p \cdot \conj{q} = j \cdot 2i \cdot (-j) = -2 j i j = -2 (-k) j =  -2i
  \]
  which corresponds to the point with coordinates $(-2,0,0)$.
  \end{example}
\end{extra}

\begin{extra}
  \begin{example}
    Quaternions represent rotations around an axis that goes through the origin $(0,0,0)$.
    What if we want to rotate around an axis that does not go through the origin?
    Say we want to rotate $P$ around an axis parallel to the Z-axis, as shown below.
    \begin{center}
      \begin{tikzpicture}
        \draw[axis] (0,-1) -- (0,4) node[at end,right] {Y};
        \draw[axis] (-1,0) -- (4,0) node[at end,above] {X};
        \point[position={(3,2)}]
        \point[position={(4,2)},label={P},anchor=west]
        \point[position={(3,3)},label={P'},anchor=south]
        \draw[-latex,thick] (4,2) arc [start angle=0,end angle=90,radius=1cm];
      \end{tikzpicture}
    \end{center}

    The achieve this, we first translate everything so that the rotation axis
    does go through (0,0,0), we perform the rotation and lastly translate everything back.

    \begin{center}
      \begin{tikzpicture}
        \draw[axis] (0,-1) -- (0,4) node[at end,right] {Y};
        \draw[axis] (-1,0) -- (4,0) node[at end,above] {X};
        \point[position={(3,2)}]
        \point[position={(4,2)},label={P},anchor=west]
        \point[position={(3,3)},label={P'},anchor=south]
        \draw[-latex,thick] (1,0) arc [start angle=0,end angle=90,radius=1cm] node[midway,font=\tiny,anchor=south west] {step 2};

        \draw[dashed,-latex] (4,2) -- (1,0) node[midway,below,sloped,font=\tiny] {step 1};
        \draw[dashed,-latex] (0,1) -- (3,3) node[midway,above,sloped,font=\tiny] {step 3};
      \end{tikzpicture}
    \end{center}

    It is certainly possible to come up with a formula that can directly deal
    with rotations around arbitrary axes (i.e.\ axes not going through the origin),
    but it would be so unwieldy that it is much simpler to work with a series of
    basic transformations (translation + rotation + translation).
  \end{example}
\end{extra}


%%% Local Variables:
%%% mode: latex
%%% TeX-master: "reference"
%%% End:

\chapter{Matrix Transformations}
\begin{Definition}[matrix] \index{matrix}
An $n \times m$ \emph{matrix} is a rectangular array of values with $N$ rows and $M$ columns.
\[
  \left[
  \begin{array}{cccc}
    a_{1,1} & a_{1,2} & \dots & a_{1,M} \\
    a_{2,1} & a_{2,2} & \dots & a_{2,M} \\
    \vdots & \vdots & \ddots & \vdots \\
    a_{N,1} & a_{N,2} & \dots & a_{N,M} \\
  \end{array}
  \right]
\]
\end{Definition}

\begin{Definition}[transpose] \index{transpose}\index{matrix!transpose}
The \emph{transpose} of a $N \times M$ matrix $a_{i,j}$ is the $M \times N$ matrix $b_{i,j}$ whose
rows are the columns of the original matrix.
\[
  b_{i,j} = a_{j,i}
\]
Notation: $\transpose{A}$.
\end{Definition}

\begin{extra}
  \begin{example}
  \[
    \transpose{\left[
      \begin{array}{ccc}
        1 & 2 & 3 \\
        4 & 5 & 6 \\
      \end{array}
    \right]} =
    \left[
      \begin{array}{cc}
        1 & 4 \\
        2 & 5 \\
        3 & 6 \\
      \end{array}
    \right]
  \]
  \end{example}
\end{extra}

\begin{Definition}[matrix multiplication] \index{matrix!multiplication}\index{multiplication!matrix}
The multiplication of an $N \times M$ matrix $a_{i,j}$ with an $M \times K$ matrix $b_{i,j}$ is defined as
the $N \times K$ matrix $c_{i,j}$ where
\[
  c_{i,j} = \sum_{x=1 \dots M} a_{i,x} b_{x,j}
\]
In other words, the element on the $i$-th row and $j$-th column
is equal to the dot product of the $i$-th row of the left matrix
and the $j$-th row of the right matrix.
\end{Definition}

\begin{extra}
  \begin{example}
  \[
    \left[
    \begin{array}{ccc}
      1 & 0 & 2 \\
      -1 & 3 & 1 \\
      0 & 1 & 2 \\
    \end{array}
    \right]
    \cdot
    \left[
    \begin{array}{ccc}
      0 & 1 & -2 \\
      -1 & 3 & 2 \\
      2 & 0 & 5
    \end{array}
    \right]
    =
    \left[
    \begin{array}{ccc}
      4 & 1 & 8 \\
      -1 & 8 & 13 \\
      3 & 3 & 12
    \end{array}
    \right]
  \]
  \end{example}
\end{extra}

\begin{Definition}[zero matrix] \index{zero matrix}\index{matrix!zero}
The \emph{zero matrix} of size $N \times M$ is the matrix where all elements are equal to zero:
$a_{1 \dots N,1 \dots M} = 0$.
\end{Definition}

\begin{Definition}[square matrix] \index{square matrix}\index{matrix!square}
A \emph{square matrix} is a matrix with an equal number of rows and columns.
\end{Definition}

\begin{Definition}[main diagonal] \index{main diagonal}\index{matrix!main diagonal}
The \emph{main diagonal} of a matrix consists of the elements $a_{i,i}$.
\[
  \left[
  \begin{array}{cccc}
    \mathbf{a_{1,1}} & a_{1,2} & a_{1,3} & \dots \\
    a_{2,1} & \mathbf{a_{2,2}} & a_{2,3} & \dots \\
    a_{3,1} & a_{3,2} & \mathbf{a_{3,3}} & \dots \\
    \vdots & \vdots & \vdots & \ddots \\
  \end{array}
  \right]
\]
\end{Definition}

\begin{Definition}[identity matrix] \index{identity matrix}\index{matrix!identity}
The \emph{identity matrix} of size $N$ is the square zero matrix of size $N \times N$
whose main diagonal contains ones. Notation: $I$.
\end{Definition}

\begin{extra}
  \begin{example}
  The identity matrix of size 4 is
  \[
    \left[
    \begin{array}{cccc}
      1 & 0 & 0 & 0 \\
      0 & 1 & 0 & 0 \\
      0 & 0 & 1 & 0 \\
      0 & 0 & 0 & 1 \\
    \end{array}
    \right]
  \]
  \end{example}
\end{extra}

\begin{theorem} \label{thm:identity-multiplication}
Multiplying a matrix with the identity matrix results in the original matrix.
\[
  A \cdot I = I \cdot A = A
\]
\end{theorem}

\begin{Definition}[inverse of a matrix] \label{def:inverse-matrix} \index{inverse}\index{matrix!inverse}
The inverse of a square matrix $A$ is the square matrix $\inverse{A}$ for which
\[
  A \cdot \inverse{A} = \inverse{A} \cdot A = I
\]
\end{Definition}

\begin{theorem}
\[
  \inverse{(A \cdot B)} = \inverse{B} \cdot \inverse{A}
\]
\end{theorem}
\begin{extra}
  \begin{proof}
  We know from \cref{def:inverse-matrix} that if
  $\inverse{B} \cdot \inverse{A}$ is to be the inverse of $A \cdot B$,
  it needs to satisfy
  \[
    (A \cdot B) \cdot (\inverse{B} \cdot \inverse{A}) = I \qquad \inverse{B} \cdot \inverse{A}) \cdot (A \cdot B) = I
  \]
  We focus on the left equality. Using associativity and \cref{thm:identity-multiplication}, we get
  \[
    (A \cdot B) \cdot (\inverse B \cdot \inverse A) =
    A \cdot (B \cdot \inverse B) \cdot \inverse A =
    A \cdot I \cdot \inverse A =
    A \cdot \inverse A =
    I
  \]
  We can prove the right equality analogously.
  \end{proof}
\end{extra}

\begin{Definition}[translation] \index{translation}
\emph{Translation} by a vector $\vec v(x,y,z)$ is represented
by the matrix
\[
  \left[
  \begin{array}{cccc}
    1 & 0 & 0 & x \\
    0 & 1 & 0 & y \\
    0 & 0 & 1 & z \\
    0 & 0 & 0 & 1 \\
  \end{array}
  \right]
\]
\end{Definition}

\begin{Definition}[applying a transformation]
To apply a transformation represented by a matrix $M$ to
a point $P(x,y,z)$, one needs to perform the following multiplication:
\[
  M \cdot \left[
            \begin{array}{c}
              x \\
              y \\
              z \\
              1 \\
            \end{array}
          \right]
\]
This yields a new matrix of the form
$\left[ \begin{array}{c} x' \\ y' \\ z' \\ 1 \end{array} \right]$.
The transformed point's coordinates are then $P'(x',y',z')$.
\end{Definition}

\begin{extra}
  \begin{example}
  Applying a translation described by $\vec v(2,-1,3)$ to a point $P(-1,3,0)$ gives
  \[
    \left[
    \begin{array}{cccc}
      1 & 0 & 0 & 2 \\
      0 & 1 & 0 & -1 \\
      0 & 0 & 1 & 3 \\
      0 & 0 & 0 & 1 \\
    \end{array}
    \right]
    \cdot
    \left[
    \begin{array}{c}
      -1 \\
      3 \\
      0 \\
      1 \\
    \end{array}
    \right]
    =
    \left[
    \begin{array}{c}
      1 \\
      2 \\
      3 \\
      1 \\
    \end{array}
    \right]
  \]
  This agrees with the result we get from $P + \vec v = (1,2,3)$.
  \end{example}
\end{extra}

\begin{extra}
  \begin{example}
  Applying a translation described by $\vec v(2,-1,0)$ to a unit circle gives
  \[
    \left[
    \begin{array}{cccc}
      1 & 0 & 0 & 2 \\
      0 & 1 & 0 & -1 \\
      0 & 0 & 1 & 0 \\
      0 & 0 & 0 & 1 \\
    \end{array}
    \right]
    \cdot
    \left[
    \begin{array}{c}
      \cos \theta \\
      \sin \theta \\
      0 \\
      1 \\
    \end{array}
    \right]
    =
    \left[
    \begin{array}{c}
      \cos(\theta) + 2 \\
      \sin(\theta) - 1 \\
      0 \\
      1 \\
    \end{array}
    \right]
  \]
  \begin{center}
    \begin{tikzpicture}
      \draw[-latex,thick] (-2,0) -- (4,0);
      \draw[-latex,thick] (0,-3) -- (0,2);

      \draw (0,0) circle (1);
      \draw[dashed] (2,-1) circle (1);
    \end{tikzpicture}
  \end{center}
  \end{example}
\end{extra}

\begin{Definition}[scaling]
\emph{Scaling} by factors $s_x$, $s_y$, $s_z$ is represented
by the matrix
\[
  \left[
  \begin{array}{cccc}
    s_x & 0 & 0 & 0 \\
    0 & s_y & 0 & 0 \\
    0 & 0 & s_z & 0 \\
    0 & 0 & 0 & 1 \\
  \end{array}
  \right]
\]
\end{Definition}

\begin{extra}
  \begin{example}
  Scaling a unit circle by $(3, \frac12,1)$ gives
  \[
    \left[
    \begin{array}{cccc}
      3 & 0 & 0 & 0 \\
      0 & \frac12 & 0 & 0 \\
      0 & 0 & 1 & 0 \\
      0 & 0 & 0 & 1 \\
    \end{array}
    \right]
    \cdot
    \left[
    \begin{array}{c}
      \cos \theta \\
      \sin \theta \\
      0 \\
      1 \\
    \end{array}
    \right]
    =
    \left[
    \begin{array}{c}
      3\cos(\theta)\\
      \frac12\sin(\theta) \\
      0 \\
      1 \\
    \end{array}
    \right]
  \]
  \begin{center}
    \begin{tikzpicture}
      \draw[-latex,thick] (-4,0) -- (4,0);
      \draw[-latex,thick] (0,-2) -- (0,2);

      \draw (0,0) circle (1);
      \draw[dashed] (0,0) circle [x radius=3cm,y radius=0.5cm];
    \end{tikzpicture}
  \end{center}
  \end{example}
\end{extra}

\begin{Definition}[rotation around X-axis] \index{rotation!X-axis}
Rotation around the X-axis by an angle $\theta$ is represented by
\[
  \left[
  \begin{array}{cccc}
    1 & 0 & 0 & 0 \\
    0 & \cos\theta & -\sin\theta & 0 \\
    0 & \sin\theta & \cos\theta & 0 \\
    0 & 0 & 0 & 1 \\
  \end{array}
  \right]
\]
\end{Definition}

\begin{Definition}[rotation around Y-axis] \index{rotation!Y-axis}
Rotation around the Y-axis by an angle $\theta$ is represented by
\[
  \left[
  \begin{array}{cccc}
    \cos\theta & 0 & \sin\theta & 0 \\
    0 & 1 & 0 & 0 \\
    -\sin\theta & 0 & \cos\theta & 0 \\
    0 & 0 & 0 & 1 \\
  \end{array}
  \right]
\]
\end{Definition}

\begin{Definition}[rotation around Z-axis] \index{rotation!Z-axis}
Rotation around the Z-axis by an angle $\theta$ is represented by
\[
  \left[
  \begin{array}{cccc}
    \cos\theta & -\sin\theta & 0 & 0 \\
    \sin\theta & \cos\theta & 0 & 0 \\
    0 & 0 & 1 & 0 \\
    0 & 0 & 0 & 1 \\
  \end{array}
  \right]
\]
\end{Definition}

\begin{extra}
  \begin{example}
  The inverse of a transformation matrix corresponds to the inverse transformation.
  Consider the translation by $(2,5,-1)$, which has transformation matrix
  \[
    \left[
    \begin{array}{cccc}
      1 & 0 & 0 & 2 \\
      0 & 1 & 0 & 5 \\
      0 & 0 & 1 & -1 \\
      0 & 0 & 0 & 1 \\
    \end{array}
    \right]  
  \]
  The inverse translation would be described by $(-2,-5,1)$ or
  \[
    \left[
    \begin{array}{cccc}
      1 & 0 & 0 & -2 \\
      0 & 1 & 0 & -5 \\
      0 & 0 & 1 & 1 \\
      0 & 0 & 0 & 1 \\
    \end{array}
    \right]  
  \]
  Multiplying these two matrices gets us
  \[
    \left[
    \begin{array}{cccc}
      1 & 0 & 0 & 2 \\
      0 & 1 & 0 & 5 \\
      0 & 0 & 1 & -1 \\
      0 & 0 & 0 & 1 \\
    \end{array}
    \right]
    \cdot
    \left[
    \begin{array}{cccc}
      1 & 0 & 0 & -2 \\
      0 & 1 & 0 & -5 \\
      0 & 0 & 1 & 1 \\
      0 & 0 & 0 & 1 \\
    \end{array}
    \right]  
    =
    \left[
    \begin{array}{cccc}
      1 & 0 & 0 & 0 \\
      0 & 1 & 0 & 0 \\
      0 & 0 & 1 & 0 \\
      0 & 0 & 0 & 1 \\
    \end{array}
    \right]  
  \]
  \end{example}
\end{extra}


%%% Local Variables:
%%% mode: latex
%%% TeX-master: "reference"
%%% End:

\chapter{Physics}
\section{Lighting Models}
\begin{Definition}[point light]
A \emph{point light} is fully defined by its position $P$ and its intensity $I$.
\end{Definition}

\begin{Definition}[ambient lighting] \index{ambient lighting}\index{lighting!ambient}
Given
\begin{itemize}
  \item an object made out of a material that is characterised by an ambient reflection coefficient $\rho_\+a$,
  \item a light ray with intensity $I$ that hits the object at a point $P$,
  \item an eye positioned at point $E$,
\end{itemize}
then the amount of light reflected by the object at $P$ toward $E$ equals
\[
  \rho_\+a \; I
\]
\end{Definition}

\begin{extra}
  \begin{example}
  We split the material's reflection coefficient into RGB-components:
  \[
    \rho_{\+a,\R} = 0.5 \qquad \rho_{\+a,\G} = 0.1 \qquad \rho_{\+a, \B} = 0.9
  \]
  Say the light has intensity
  \[
    I_\R = 1 \qquad I_\G = 0.5 \qquad I_\B = 0
  \]
  The material then reflects light with intensity (0.5, 0.05, 0).
  \end{example}
\end{extra}

\begin{Definition}[diffuse lighting] \index{diffuse lighting}\index{lighting!diffuse}
Given
\begin{itemize}
  \item an object made of a material that is characterised by a diffuse reflection coefficient $\rho_\+d$,
  \item a light ray that hits an object at position $P$,
  \item an eye positioned at point $E$,
\end{itemize}
then the amount of light reflected toward $E$ equals
\[
  \rho_\+d \; I \; \cos\theta
\]
where $\theta$ is the angle between the normal vector on the surface of the object at $P$ and
the incoming light ray.
\begin{center}
  \begin{tikzpicture}
    \draw[thick] (0,0) circle (1);
    \draw[vector] (30:1) -- (30:2) node[at end,anchor=south] {$\vec n$};
    \draw[light] (30:1) +(60:1) -- (30:1);
    \draw (30:1.5) arc[start angle=30,end angle=60,radius=0.5] node[midway,font=\tiny,anchor=south west,inner sep=1mm] {$\theta$};
  \end{tikzpicture}
\end{center}
\end{Definition}

\begin{extra}
  \begin{example}
  \begin{center}
    \begin{tikzpicture}
      \coordinate (L) at (1,3);
      \coordinate (H) at (0,0);
      \coordinate (N) at (45:1);

      \draw[thick] (-2,2) -- (2,-2);
      \draw[vector] (H) -- (N) node[above] {$\vec n$};
      \point[/point/label=L,/point/position=(L)]
      \point[/point/label=H,/point/anchor=north east,/point/position=(H)]

      \draw[light] (L) -- (H);
      \tikzmath{
        real \a;
        \a = atan2(3,1);
      }
      \draw (45:0.5) arc [start angle=45,end angle=\a,radius=0.5cm,font=\tiny] node[anchor=south west,inner sep=1pt,midway] {$\scriptstyle\theta$};
    \end{tikzpicture}
  \end{center}
  with
  \[
    \begin{array}{r@{\;}c@{\;}l}
      I & = & \colorrgb{1}{0.75}{0} \\
      \rho_\+d & = & \colorrgb{1}{0.5}{1} \\[2mm]
      L & = & (1,3,0) \\
      H & = & (0,0,0) \\
      \vec n & = & (0.707107,0.707107,0) \\
    \end{array}
  \]
  The angle $\theta$ can be found with
  \[
    \cos\theta = \vec n \cdot \frac{L-H}{\norm{L-H}} = 0.894
  \]
  which means that due to the incident angle the object will
  be illuminated at 89.4\% of the light's total brightness.
  Of these 89.4\%, only part is reflected by the object's material.
  Exactly how much is determined is determined by $\rho_\+d$.
  \[
    I \cdot \rho_\+d \cdot \cos\theta = (1 \cdot 1 \cdot 0.894, 0.75 \cdot 0.5 \cdot 0.894, 0 \cdot 1 \cdot 0.894) = (0.894,0.335,0)
  \]
  \end{example}
\end{extra}

\begin{Definition}[specular lighting] \index{specular lighting}\index{lighting!specular}
Given
\begin{itemize}
  \item an object made of a material that is characterised by  a specular reflection
        coefficient $\rho_\+d$ and a phong exponent $e$,
  \item a light ray that hits an object at position $P$,
  \item an eye positioned at point $E$,
\end{itemize}
then the amount of light reflected toward $E$ equals
\[
  \rho_\+s \; I \; (\cos\theta)^e
\]
where $\theta$ is the angle between the reflection of the light ray on the surface of the object at $P$ and
the vector $\overrightarrow{PE}$.
\begin{center}
  \begin{tikzpicture}
    \coordinate (P) at (30:1);
    \coordinate (eye) at ($ (P) + (-20:2) $);
    \draw[thick] (0,0) circle (1);
    \draw[vector] (P) -- (30:2) node[at end,anchor=south] {$\vec n$};
    \draw[light] (P) +(60:1) -- (P) -- ++(0:1);
    \point[/point/label=E,/point/position=(eye)];
    \draw (P) -- (eye);
    \draw (30:1) ++ (0.5,0) arc[start angle=0,end angle=-20,radius=0.5] node[midway,font=\tiny,anchor=west,inner sep=1mm] {$\theta$};
  \end{tikzpicture}
\end{center}
\end{Definition}

\begin{extra}
  \begin{example}
  \begin{center}
    \begin{tikzpicture}
      \coordinate (L) at (1,3);
      \coordinate (R) at (3,1);
      \coordinate (H) at (0,0);
      \coordinate (N) at (45:1);
      \coordinate (E) at (6,0);

      \draw[thick] (-2,2) -- (2,-2);
      \draw[vector] (H) -- (N) node[above] {$\vec n$};
      \point[/point/label=L,/point/position=(L)]
      \point[/point/label=H,/point/anchor=north east,/point/position=(H)]
      \point[/point/label=E,/point/position=(E)]

      \draw[light] (L) -- (H) -- (R);
      \draw (H) -- (E);
      \tikzmath{
        real \a;
        \a = atan2(1,3);
      }
      \draw (0:1) arc [start angle=0,end angle=\a,radius=1cm,font=\tiny] node[anchor=west,inner sep=1pt,midway] {$\scriptstyle\theta$};
    \end{tikzpicture}
  \end{center}
  with
  \[
    \begin{array}{r@{\;}c@{\;}l@{\qquad}r@{\;}c@{\;}l}
      I & = & \colorrgb{1}{0.75}{0} & L & = & (1,3,0) \\
      \rho_\+s & = & \colorrgb{1}{0.5}{1} & H & = & (0,0,0) \\
      e & = & 10 & E & = & (6,0,0) \\
      &&& \vec n & = & (0.707107,0.707107,0) \\
    \end{array}
  \]
  We compute how the light ray is reflected. We reflected vector is $H-L$, the reflecting vector is $\vec n$:
  \[
    \vec{r} = v - 2 \cdot (\vec v \cdot \vec n) \cdot n = (3,1,0)
  \]
  We need the angle between $\vec r$ and $E-H$:
  \[
    \cos\theta=\frac{\vec r}{\norm{\vec r}} \cdot \frac{E-H}{\norm{E-H}} = 0.949
  \]
  We compute the final colour:
  \[
    I \cdot \rho_s \cdot \cos(\theta)^e = (0.592464, 0.222174, 0)
  \]
  \end{example}
\end{extra}


\section{Refraction}
\begin{Definition}[Snell's law] \index{Snell's law} \index{refractive index}
Given a light ray that moves from a medium with refractive index $n_1$
into a medium with refractive index $n_2$, the incoming angle $\theta_\+i$
and $\theta_\+o$ obey the following law:
\[
  n_1 \sin \theta_\+i = n_2 \sin \theta_\+o
\]
\begin{center}
  \begin{tikzpicture}
    \draw[ultra thick] (-2,0) -- (2,0);
    \draw[dotted] (0,-2) -- (0,2);
    \draw[vector] (120:2) -- (0,0);
    \draw[vector] (0,0) -- (-80:2);
    \draw (90:0.5) arc [start angle=90,end angle=120,radius=0.5] node[font=\tiny,anchor=south,midway] { $\theta_\+i$ };
    \draw (-80:0.5) arc [start angle=-80,end angle=-90,radius=0.5] node[font=\tiny,anchor=north east,midway] { $\theta_\+o$ };
    \node at (1,1) {$n_1$};
    \node at (1,-1) {$n_2$};
  \end{tikzpicture}
\end{center}
\end{Definition}

\begin{Definition}[refractive indices] \index{refractive index}
The following materials' refractive indices are listed in the table below:
\begin{center}
  \begin{tabular}{rl}
    {\bfseries Material} & {\bfseries Refractive index} \\
    \toprule
    Air \index{air} & 1 \\
    Water \index{water} & 1.33 \\
    Glass \index{glass} & 1.5 \\
    Sapphire \index{sapphire} & 1.77 \\
    Diamond \index{diamond} & 2.42 \\
  \end{tabular}
\end{center}
\end{Definition}

\begin{Definition}[total internal reflection] \index{total internal reflection}
\emph{Total internal reflection} occurs when
\[
  \frac{n_1}{n_2} \sin \theta_\+i > 1
\]
In this case, the light ray is not refracted but reflected.
\end{Definition}

\begin{samepage}
\begin{theorem}\label{thm:vectorial-refraction}
Given
\begin{itemize}
  \item a light ray that moves in a direction described by the unit vector $\vec v$, and
  \item this light ray moves from a medium with refractive index $n_1$ into a medium with refractive index $n_2$, and
  \item $\vec n$ is the normal vector on the surface separating both mediums at the position where
        the light ray hits this surface,
\end{itemize}
\begin{center}
  \begin{tikzpicture}
    \draw[ultra thick] (-2.5,0) -- (2.5,0);
    \draw[dotted] (0,-2) -- (0,2);
    \draw[vector] (135:2) -- (0,0) node[midway,anchor=south west,font=\tiny] {$\vec v$};
    \draw[vector] (0,0) -- (-70:2) node[midway,anchor=south west,font=\tiny] {$\vec r$};
    \draw[vector] (0,0) -- (90:2) node[midway,anchor=south west,font=\tiny] {$\vec n$};
    \node at (2,1) {$n_1$};
    \node at (2,-1) {$n_2$};
    \draw[dotted,thin] (0,0) circle (2);
  \end{tikzpicture}
\end{center}
Then the direction of the refracted light ray can be found by computing
\[
  \begin{array}{r@{\;}c@{\;}l}
    \vec r_\+x & = & \displaystyle \frac{n_1}{n_2} (\vec v - (\vec v \cdot \vec n) \cdot \vec n) \\ \\
    \vec r_\+y & = & -\sqrt{1-\norm{\vec r_\+x}^2} \cdot \vec n \\ \\
    \vec r & = & \vec r_\+x + \vec r_\+y
  \end{array}
\]
\end{theorem}
\end{samepage}
\begin{extra}
  \begin{proof}
  We add a few elements to the sketch:
  \begin{center}
    \begin{tikzpicture}
      \draw[ultra thick] (-2,0) -- (2,0);
      \draw[dotted] (0,-2) -- (0,2);
      \draw[vector] (135:2) -- (0,0) node[midway,anchor=south west,font=\tiny] {$\vec v$};
      \draw[vector,thin,red] let \p1=(135:2) in (\p1) -- (0,\y1) node[midway,anchor=south,font=\tiny] {$\vec v_\+x$};
      \draw[vector,thin,red] let \p1=(135:2) in (\p1) -- (\x1,0) node[midway,anchor=east,font=\tiny] {$\vec v_\+y$};
      \draw[vector] (0,0) -- (-70:2) node[midway,anchor=south west,font=\tiny] {$\vec r$};
      \draw (90:0.5) arc [start angle=90,end angle=135,radius=0.5] node[font=\tiny,anchor=south,midway] { $\theta_\+i$ };
      \draw (-70:0.5) arc [start angle=-70,end angle=-90,radius=0.5] node[font=\tiny,anchor=north,midway,xshift=1pt] { $\theta_\+o$ };
      \draw[vector,thin,red] let \p1=(-70:2) in (0,0) -- (\x1,0) node[midway,anchor=south,font=\tiny] {$\vec r_\+x$};
      \draw[vector,thin,red] let \p1=(-70:2) in (0,0) -- (0,\y1) node[midway,anchor=east,font=\tiny] {$\vec r_\+y$};
      \node at (2,1) {$n_1$};
      \node at (2,-1) {$n_2$};
    \end{tikzpicture}
  \end{center}
  From the sketch, we can deduce
  \[
    \begin{array}{r@{\;}c@{\;}l@{\hspace{1cm}}r@{\;}c@{\;}l}
      \norm{\vec v} & = & 1    & \norm{\vec r} & = & 1 \\ \\
      \vec v & = & \vec v_\+x + \vec v_\+y    & \vec r & = & \vec r_\+x + \vec r_\+y \\ \\
      \sin \theta_\+i & = & \displaystyle \frac{\norm{\vec v_\+x}}{\norm{\vec v}} & \sin \theta_\+o & = & \displaystyle \frac{\norm{\vec r_\+x}}{\norm{\vec r}} \\ \\
    \end{array}
  \]
  \[
    n_1 \sin \theta_\+i = n_2 \sin \theta_\+o
  \]
  Our goal is to express $\vec{r}$ in terms of $\vec v$, $\vec n$, $n_1$ and $n_2$.
  We decompose $\vec v$ into $\vec v_\+x$ and $\vec v_\+y$:
  \[
    \vec v_\+y = (\vec v \cdot \vec n) \cdot \vec n \qquad \vec v_\+x = \vec v-\vec v_\+y = \vec v - (\vec v \cdot \vec n) \cdot \vec n
  \]
  We determine $\vec r_\+x$'s length:
  \[
    \norm{\vec r_\+x} = \norm{\vec r} \sin\theta_\+o = \sin\theta_\+o = \frac{n_1}{n_2} \sin \theta_\+i = \frac{n_1}{n_2} \norm{\vec v_\+x}
  \]
  Since $\vec{r}_\+x$ is parallel to $\vec{v}_\+x$ and points in the same direction, we get
  \[
    \vec r_\+x = \frac{n_1}{n_2} \vec v_\+x = \frac{n_1}{n_2} (\vec v - (\vec v \cdot \vec n) \cdot \vec n)
  \]
  We now turn our attention to $\vec r_\+y$.
  \[
    \norm{\vec r_\+x}^2 + \norm{\vec r_\+y}^2 = 1 \qquad\Rightarrow\qquad \norm{\vec r_\+y} = \sqrt{1 - \norm{\vec r_\+x}^2}
  \]
  $\vec r_\+y$ is parallel to $\vec n$:
  \[
    \vec r_\+y = - \sqrt{1 - \norm{\vec r_\+x}^2} \cdot \vec n
  \]
  \end{proof}
\end{extra}

\begin{extra}
  \begin{example}
  \begin{center}
    \begin{tikzpicture}
      \draw[ultra thick] (-2,0) -- (2,0);
      \draw[dotted] (0,-2) -- (0,2);
      \draw[vector] (135:2) -- (0,0) node[midway,anchor=south west,font=\tiny] {$\vec v$};
      \draw[vector] (0,0) -- (-70:2) node[midway,anchor=south west,font=\tiny] {$\vec r$};
      \draw (90:0.5) arc [start angle=90,end angle=135,radius=0.5] node[font=\tiny,anchor=south,midway] { $45\degrees$ };
      \draw (-70:0.5) arc [start angle=-70,end angle=-90,radius=0.5] node[font=\tiny,anchor=north,midway,xshift=1pt] { $\theta_\+o$ };
      \node at (2,1) {$n_1 = 1$};
      \node at (2,-1) {$n_2 = 2$};
    \end{tikzpicture}
  \end{center}
  We first compute the outgoing angle $\theta_\+o$ using Schnell's law:
  \[
    \sin 45\degrees = 2 \sin \theta_\+o \qquad\Rightarrow\qquad \theta_\+o = \arcsin\left(\frac{\sin 45\degrees}{2}\right) = 20.7 \degrees
  \]
  We determine $\vec v$'s and $\vec r$'s coordinates:
  \[
    \begin{array}{r@{\;}c@{\;}l@{\;}c@{\;}l}
      \vec v & = & (\cos 45\degrees, -\sin 45\degrees) & = & (0.707, -0.707) \\
      \vec r & = & (\cos (90\degrees-20.7\degrees), -\sin (90\degrees-20.7\degrees)) & = & (0.353, -0.935)
    \end{array}
  \]
  We now check whether the result using \cref{thm:vectorial-refraction} agrees with this.
  \[
    \begin{array}{r@{\;}c@{\;}l@{\;}c@{\;}l}
      \vec r_\+x & = & \displaystyle \frac{n_1}{n_2} (\vec v - (\vec v \cdot \vec n) \cdot \vec n) & = & (0.353,0) \\ \\
      \vec r_\+y & = & -\sqrt{1-\norm{\vec r_\+x}^2} \cdot \vec n & = & (0, -0.935) \\ \\
      \vec r & = & \vec r_\+x + \vec r_\+y & = & (0.353, -0.935)
    \end{array}
  \]
  \end{example}
\end{extra}

%%% Local Variables:
%%% mode: latex
%%% TeX-master: "reference"
%%% End:

\chapter{SDL}
\begin{center}
  \newcommand{\tag}[3][2mm]{
    {\tt #2} & #3 \\
  }
  \renewcommand{\arraystretch}{1.5}
  \begin{tabular}{ll}
    \textbf{Tag} & \textbf{Effect} \\
    \toprule
    \tag{background $r$ $g$ $b$}{Sets the background colour}
    \tag{ambient $r$ $g$ $b$}{Sets the current ambient colour}
    \tag{diffuse $r$ $g$ $b$}{Sets the current diffuse colour}
    \tag{specular $r$ $g$ $b$ $e$}{Sets the current specular colour}
    \tag{reflectivity $r$}{Sets the reflectivity}
    \tag{transparency $t$}{Sets the transparency}
    \tag{refractiveIndex $i$}{Sets the refractive index}
    \tag{light $x$ $y$ $z$ $r$ $g$ $b$}{Adds a light with given position and colour}
    \midrule
    \tag{sphere}{Sphere centred at $(0,0,0)$ with radius $1$}
    \tag{square}{Square centred at $(0,0,0)$ with sides $2$ lying XY plane}
    \tag{tetrahedron}{Adds a tetrahedron}
    \tag{fsphere $n$}{Adds a faceted sphere made of $4^n$ triangles}
    \tag{mesh \textit{filename}}{Adds a mesh}
    \midrule
    \tag{scale $s_x$ $s_y$ $s_z$}{Prepends a scaling transformation}
    \tag{translate $d_x$ $d_y$ $d_z$}{Prepends a translation}
    \tag{rotateX $theta$}{Prepends a rotation around the X axis}
    \tag{rotateY $theta$}{Prepends a rotation around the Y axis}
    \tag{rotateZ $theta$}{Prepends a rotation around the Z axis}
    \tag{push}{Pushes a copy of the current transformation onto the stack}
    \tag{pop}{Pops a transformation from the stack}
    \bottomrule
  \end{tabular}
\end{center}



%%% Local Variables:
%%% mode: latex
%%% TeX-master: "reference"
%%% End:

\chapter{Raytracer Design}

\section{Shape Hierarchy}
\begin{center}
  \begin{tikzpicture}
    \node[/uml/class] (raytraceable) {
      \umlclass\Abstract Raytraceable%
      \Fields%
      \Methods%
      \Public\Abstract hits(ray, tMax) :\ bool \\
      \Public\Abstract intersection(ray, hitInfo) : bool \\
      \endumlclass
    };

    \node[/uml/class,anchor=east] (shape) at ($ (raytraceable.south) + (-1,-3) $) {
      \umlclass\Abstract Shape%
      \Fields%
      \Public material \\
      \Methods%
      \endumlclass
    };

    \node[/uml/class,anchor=north east] (sphere) at ($ (shape.south) + (-1,-1) $) {
      \umlclass Sphere%
      \Fields%
      \Methods%
      \endumlclass
    };

    \node[/uml/class,anchor=north east] (square) at ($ (sphere.south east) + (0,-1) $) {
      \umlclass Square%
      \Fields%
      \Methods%
      \endumlclass
    };

    \node[/uml/class,anchor=north east] (tetrahedron) at ($ (square.south east) + (0,-1) $) {
      \umlclass Tetrahedron%
      \Fields%
      \Methods%
      \endumlclass
    };

    \node[/uml/class,anchor=west] (mesh) at ($ (sphere.east) + (2,0) $) {
      \umlclass Mesh%
      \Fields%
      \Methods%
      \endumlclass
    };

    \node[/uml/class,anchor=west] (faceted-sphere) at ($ (square.east) + (2,0) $) {
      \umlclass FacetedSphere%
      \Fields%
      \Methods%
      \endumlclass
    };

    \node[/uml/class,anchor=west] (face) at ($ (raytraceable.south) + (2,-3) $) {
      \umlclass Face%
      \Fields%
      \Methods%
      \endumlclass
    };

    \node[/uml/class,anchor=north west] (polygonal-face) at ($ (face.south) + (1,-1) $) {
      \umlclass PolygonalFace%
      \Fields%
      \Methods%
      \endumlclass
    };

    \node[/uml/class,anchor=north west] (flat-polygonal-face) at ($ (polygonal-face.south) + (1,-1) $) {
      \umlclass FlatPolygonalFace%
      \Fields%
      \Methods%
      \endumlclass
    };

    \node[/uml/class,anchor=north west] (smooth-triangle-face) at ($ (flat-polygonal-face.south west) + (0,-1) $) {
      \umlclass SmoothTriangleFace%
      \Fields%
      \Methods%
      \endumlclass
    };

    \node[/uml/class,anchor=east] (union) at ($ (raytraceable.south) + (-3,-1) $) {
      \umlclass Union%
      \Fields%
      \Methods%
      \endumlclass
    };

    \node[/uml/class,anchor=west] (transformer) at ($ (raytraceable.south) + (3,-1) $) {
      \umlclass Transformer%
      \Fields%
      \Public transformation \\    
      \Methods%
      \endumlclass
    };

    \draw[/uml/inherits] (union.east) -| (raytraceable.south);
    \draw[/uml/association] (union.north) |- (raytraceable.west) node[anchor=south east] {0..*};
    \draw[/uml/inherits] (shape.east) -| (raytraceable.south);
    \draw[/uml/inherits] (face.west) -| (raytraceable.south);
    \draw[/uml/inherits] (transformer.west) -| (raytraceable.south);
    \draw[/uml/association] (transformer.north) |- (raytraceable.east) node[anchor=south west] {1};
    \draw[/uml/inherits] (sphere.east) -| (shape.south);
    \draw[/uml/inherits] (square.east) -| (shape.south);
    \draw[/uml/inherits] (tetrahedron.east) -| (shape.south);
    \draw[/uml/inherits] (mesh.west) -| (shape.south);
    \draw[/uml/inherits] (faceted-sphere.west) -| (shape.south);
    \draw[/uml/inherits] (polygonal-face.west) -| (face.south);
    \draw[/uml/inherits] (flat-polygonal-face.west) -| (polygonal-face.south);
    \draw[/uml/inherits] (smooth-triangle-face.west) -| (polygonal-face.south);
  \end{tikzpicture}
\end{center}


\section{Renderer Hierarchy}
\begin{center}
  \begin{tikzpicture}
    \node[/uml/class] (renderer) {
      \umlclass\Abstract Renderer%
      \Fields%
      \Private camera; \\
      \Private rayTracer; \\
      \Methods%
      \Public\Abstract renderPixel :\ Colour \\
      \Public\Abstract render(image) \\
      \endumlclass
    };

    \node[/uml/class,anchor=north east] (st-renderer) at ($ (renderer.south) + (-1,-1) $) {
      \umlclass\Abstract SingleThreadedRenderer%
      \Fields%
      \Methods%
      \endumlclass
    };

    \node[/uml/class,anchor=north east] (mt-renderer) at ($ (st-renderer.south east) + (0,-1) $) {
      \umlclass\Abstract MultiThreadedRenderer%
      \Fields%
      \Methods%
      \endumlclass
    };

    \node[/uml/class,anchor=west] (jaggy-renderer) at ($ (st-renderer.east) + (2,0) $) {
      \umlclass\Abstract JaggyRenderer%
      \Fields%
      \Methods%
      \endumlclass
    };

    \node[/uml/class,anchor=west] (aa-renderer) at ($ (mt-renderer.east) + (2,0) $) {
      \umlclass\Abstract AntiAliasingRenderer%
      \Fields%
      \Methods%
      \endumlclass
    };

    \draw[/uml/inherits] (st-renderer.east) -| (renderer.south);
    \draw[/uml/inherits] (mt-renderer.east) -| (renderer.south);
    \draw[/uml/inherits] (jaggy-renderer.west) -| (renderer.south);
    \draw[/uml/inherits] (aa-renderer.west) -| (renderer.south);
  \end{tikzpicture}
\end{center}

\section{RayTracer Hierarchy}
\begin{center}
  \begin{tikzpicture}
    \node[/uml/class] (raytracer) {
      \umlclass\Abstract RayTracer%
      \Fields%
      \Private scene; \\
      \Methods%
      \Public\Abstract shade(ray, weight) :\ Colour \\
      \Public\Abstract render(image, hitInfo, weight) : Colour \\
      \endumlclass
    };

    \node[/uml/class,anchor=north] (smoothshading-raytracer) at ($ (raytracer.south) + (0,-1) $) {
      \umlclass SmoothShadingRayTracer%
      \Fields%
      \Methods%
      \endumlclass
    };

    \node[/uml/class,anchor=north] (shadowing-smoothshading-raytracer) at ($ (smoothshading-raytracer.south) + (0,-1) $) {
      \umlclass ShadowingSmoothShadingRayTracer%
      \Fields%
      \Methods%
      \endumlclass
    };

    \node[/uml/class,anchor=north] (reflecting-shadowing-smoothshading-raytracer) at ($ (shadowing-smoothshading-raytracer.south) + (0,-1) $) {
      \umlclass ReflectingShadowingSmoothShadingRayTracer%
      \Fields%
      \Methods%
      \endumlclass
    };

    \node[/uml/class,anchor=north] (refractive-reflecting-shadowing-smoothshading-raytracer) at ($ (reflecting-shadowing-smoothshading-raytracer.south) + (0,-1) $) {
      \umlclass RefractiveReflectingShadowingSmoothShadingRayTracer%
      \Fields%
      \Methods%
      \endumlclass
    };

    \draw[/uml/inherits] (smoothshading-raytracer.north) -- (raytracer.south);    
    \draw[/uml/inherits] (shadowing-smoothshading-raytracer.north) -- (smoothshading-raytracer.south);    
    \draw[/uml/inherits] (reflecting-shadowing-smoothshading-raytracer.north) -- (shadowing-smoothshading-raytracer.south);    
    \draw[/uml/inherits] (refractive-reflecting-shadowing-smoothshading-raytracer.north) -- (reflecting-shadowing-smoothshading-raytracer.south);    
  \end{tikzpicture}
\end{center}

%%% Local Variables:
%%% mode: latex
%%% TeX-master: "reference"
%%% End:


\begin{extra}
  \chapter{Exam Questions}
\section{Mathematics}
\begin{Exercise}
Say we wish to rotate an arbitrary point $(x,y,z)$ over an angle $\theta$
around the point $(2,1,3)$ in the plane XZ.
Give two ways to compute this.
\begin{solution}
\begin{itemize}
  \item Using quaternions. Rotated point has coordinates $(x' + 2, y' + 1, z' + 3)$:
        \[
          \begin{array}{rcl}
            q & = & (x - 2) i+(y - 1) j+(z - 3) k \\ \\
            r & = & \displaystyle \cos\left(\frac\theta2\right) +
                    \sin\left(\frac\theta2\right) j \\ \\
            \conj{r} & = & \displaystyle \cos\left(\frac\theta2\right) -
                           \sin\left(\frac\theta2\right) j \\ \\
            x'i+y'j+z'k & = & r \cdot q \cdot \conj{r} \\
          \end{array}
        \]
  \item Using matrices
        \[
          \left[
            \begin{array}{c}
              x' \\
              y' \\
              z' \\
              1
            \end{array}
          \right]
          =
          \left[
            \begin{array}{cccc}
              1 & 0 & 0 & 2 \\
              0 & 1 & 0 & 1 \\
              0 & 0 & 1 & 3 \\
              0 & 0 & 0 & 1 \\
            \end{array}
          \right]
          \cdot
          \left[
            \begin{array}{cccc}
              \cos\theta & 0 & \sin\theta & 0 \\
              0 & 1 & 0 & 0 \\
              -\sin\theta & 0 & \cos\theta & 0 \\
              0 & 0 & 0 & 1 \\
            \end{array}
          \right]
          \cdot
          \left[
            \begin{array}{cccc}
              1 & 0 & 0 & -2 \\
              0 & 1 & 0 & -1 \\
              0 & 0 & 1 & -3 \\
              0 & 0 & 0 & 1 \\
            \end{array}
          \right]
          \cdot
          \left[
            \begin{array}{c}
              x \\
              y \\
              z \\
              1
            \end{array}
          \right]
        \]
\end{itemize}
\end{solution}
\end{Exercise}


\begin{Exercise}
I want to be able to rotate an arbitrary point $(x,y,z)$ around a point $(4,0,3)$ in a plane
with normal vector $(1,2,3)$ by some angle $\alpha$. What computations do I need to perform?
\end{Exercise}


\begin{Exercise}
Which transformation matrices transform the solid shape $S_1$ into the dashed shape $S_2$ and back?
In other words, find the matrices $A$ and $B$ such that
\[
  A \cdot S_1 = S_2 \qquad B \cdot S_2 = S_1
\]
Write $A$ and $B$ as multiplications of basic transformation matrices (translation, scaling, rotation).
The grid is tilted 45\degrees. Use it to help you determine the exact size of the shapes.
\begin{center}
  \begin{tikzpicture}
    \path[clip] (-5,-3) rectangle (5,4);
    \draw[axis] (-4,0) -- (4,0) node[anchor=south] {X};
    \draw[axis] (0,-4) -- (0,4) node[anchor=north west] {Y};
    \draw[thin,gray,rotate=45] (-8,-8) grid (8,8);

    \draw[thick] (0,0) circle (1);

    \draw[dashed,thick,rotate=45] (0,1) circle [x radius=2,y radius=3];
  \end{tikzpicture}
\end{center}
\begin{solution}
\[
  M =
  \left[
    \begin{array}{ccc}
      \cos(-45\degrees) & -\sin(-45\degrees) & 0 \\
      \sin(-45\degrees) & \cos(-45\degrees) & 0 \\
      0 & 0 & 1
    \end{array}
  \right]
  \cdot
  \left[
    \begin{array}{ccc}
      1 & 0 & -1 \\
      0 & 1 & 0 \\
      0 & 0 & 1
    \end{array}
  \right]
  \cdot
  \left[
    \begin{array}{ccc}
      3 & 0 & 0 \\
      0 & 2 & 0 \\
      0 & 0 & 1
    \end{array}
  \right]
\]
\[
  M^{-1} =
  \left[
    \begin{array}{ccc}
      1/3 & 0 & 0 \\
      0 & 1/2 & 0 \\
      0 & 0 & 1
    \end{array}
  \right]
  \cdot
  \left[
    \begin{array}{ccc}
      1 & 0 & 1 \\
      0 & 1 & 0 \\
      0 & 0 & 1
    \end{array}
  \right]
  \cdot
  \left[
    \begin{array}{ccc}
      \cos(45\degrees) & -\sin(45\degrees) & 0 \\
      \sin(45\degrees) & \cos(45\degrees) & 0 \\
      0 & 0 & 1
    \end{array}
  \right]
\]
\end{solution}
\end{Exercise}

\begin{Exercise}
Which transformation matrices transform the solid shape $S_1$ into the dashed shape $S_2$ and back?
In other words, find the matrices $A$ and $B$ such that
\[
  A \cdot S_1 = S_2 \qquad B \cdot S_2 = S_1
\]
Write $A$ and $B$ as multiplications of basic transformation matrices (translation, scaling, rotation).
The grid is tilted 30\degrees. Use it to help you determine the exact size of the shapes.
\begin{center}
  \begin{tikzpicture}
    \path[clip] (-5,-5) rectangle (5,5);
    \draw[axis] (-4,0) -- (4,0) node[anchor=south] {X};
    \draw[axis] (0,-4) -- (0,4) node[anchor=west] {Y};
    \draw[thin,gray,rotate=30] (-8,-8) grid (8,8);

    \begin{scope}[rotate=30]
      \draw[thick] (1,0) rectangle ++(2,1);
    \end{scope}

    \begin{scope}[rotate=120]
      \draw[thick,dashed] (1,0) rectangle ++(1,1);
    \end{scope}
  \end{tikzpicture}
\end{center}
\begin{solution}
\[
  M = \mathrm{Rot}[30\degrees] \cdot \mathrm{Tr}[-1,1] \cdot \mathrm{Sc}[\frac12,1] \cdot \mathrm{Tr}[-1,0] \cdot \mathrm{Rot}[-30\degrees]
\]
\[
  M^{-1} = \mathrm{Rot}[30\degrees] \cdot \mathrm{Tr}[1,0] \cdot \mathrm{Sc}[2,1] \cdot \mathrm{Tr}[1,-1] \cdot \mathrm{Rot}[-30\degrees]
\]
Has to be written down in matrix form on the exam, but using shorthand notation first can help.
\end{solution}

\end{Exercise}


\section{Physics}

\begin{Exercise}
Compute the lighting at position $P$ as seen from $E$. Take into account ambient, diffuse and specular lighting.
\begin{center}
  \begin{tikzpicture}
    \path[use as bounding box,clip] (-3,-2) rectangle (3,3);
    \draw[thin,gray] (-5,-5) grid (5,5);
    \coordinate (P) at (0,0);
    \coordinate (L) at (-1,2);
    \coordinate (E) at (2,1);
    \draw[thick] (0,-4) circle (4);
    \point[/point/label=P,/point/position=(P)]
    \point[/point/label=L,/point/position=(L)]
    \point[/point/label=E,/point/position=(E)]
  \end{tikzpicture}
\end{center}
\begin{center}
  \begin{tabular}{lcccc}
    & \textbf{r} & \textbf{g} & \textbf{b} \\
    \toprule
    light source & 1 & 1 & 0 \\
    ambient & 0.2 & 0.2 & 0.2 \\
    diffuse & 0.8 & 0.0 & 0.9 \\
    specular & 0.5 & 0.5 & 0.5 & e = 10
  \end{tabular}
\end{center}
\end{Exercise}


\begin{Exercise}
A sphere is centred at $C(3, 0, 0)$ and has radius $2$.
A ray of light starts at $P(2,2,2)$ and is directed straight towards
the sphere's centre. It hits the sphere at some position $H$.
The ray bounces off the sphere in a mirror-like way and proceeds to hit the YZ-plane at some point $Q$.
\begin{itemize}
  \item What are $H$'s coordinates?
  \item What is the reflected ray's direction?
  \item What are $Q$'s coordinates?
\end{itemize}
\end{Exercise}

\begin{Exercise}
A light ray traverses a piece of glass.
\begin{center}
  \begin{tikzpicture}
    \draw[thick] (0,0) -- ++(0,2);
    \draw[thick] (5,0) -- ++(0,2);
    \draw[light] (-1,2) -- (0,1.5) -- (5,0.5) -- (6,0);

    \draw[|-|,thin] (0,-.5) -- ++(5,0) node[below,midway] {5};
    \draw[dashed,thin] (0,1.5) -- (6,1.5);
    \draw[dashed,thin] (5,0.5) -- (6,0.5);
    \draw[|-|,thin] (6.5,0.5) -- (6.5,1.5) node[right,midway] {x};

    \node at (-2,0) {air};
    \node at (2.5,0) {glass};
    \node at (7,0) {air};
  \end{tikzpicture}
\end{center}
\end{Exercise}
The direction of the light is $(2,-1)$, the thickness of the glass is 5. What is vertical distance travelled by the light ray, represented by $x$?

\section{Ray Tracer}
\begin{Exercise}
Give the SDL file that produces this result:
\begin{center}
  \includegraphics[width=.5\linewidth]{scene.png}
\end{center}
The camera is located at $(0,0,-5)$, looks at $(0,0,0)$ and has up-vector $(0,1,0)$.
The sphere in the middle has radius 1; the lower sphere is twice as big.
Make sure not to forget about the following details:
\begin{itemize}
  \item The lights
  \item The material's properties (these do not have to be 100\% correct)
  \item The exact transformations.
\end{itemize}
\end{Exercise}

\section{Extensions}
\begin{Exercise}
Currently, our ray tracer allows us to specify only one colour to each shape:
a shape is either completely red, or blue, or white, or \dots
Say we want to achieve the following:
\begin{center}
  \includegraphics[width=.5\linewidth]{3d-textures.png}
\end{center}
What changes would you make to the ray tracer?
\end{Exercise}

\begin{Exercise}
Our shapes are perfectly smooth. It might come in handy to give them a rougher surface.
Take for example the bumpy sphere shown below.
\begin{center}
  \includegraphics[width=.5\linewidth]{bumpmapping.png}
\end{center}
What changes would you make to the ray tracer?
\end{Exercise}

\begin{Exercise}
Our ray tracer understands unions of objects, i.e. it is possible to create
a new shape by grouping two shapes together. There are other operations
like this, which yield more interesting results. Below you can see the result
of \emph{intersecting} two spheres to produce a lens.
\begin{center}
  \includegraphics[width=.5\linewidth]{csg.png}
\end{center}
What changes would you make to support intersection?
\end{Exercise}


%%% Local Variables:
%%% mode: latex
%%% TeX-master: "reference"
%%% End:

\end{extra}

\printindex

\end{document}



%%% Local Variables: 
%%% mode: latex
%%% TeX-master: t
%%% End: 
