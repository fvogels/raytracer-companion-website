\section{Extensions}
\begin{Exercise}
Currently, our ray tracer allows us to specify only one colour to each shape:
a shape is either completely red, or blue, or white, or \dots
Say we want to achieve the following:
\begin{center}
  \includegraphics[width=.5\linewidth]{3d-textures.png}
\end{center}
What changes would you make to the ray tracer?
\end{Exercise}

\begin{Exercise}
Our shapes are perfectly smooth. It might come in handy to give them a rougher surface.
Take for example the bumpy sphere shown below.
\begin{center}
  \includegraphics[width=.5\linewidth]{bumpmapping.png}
\end{center}
What changes would you make to the ray tracer?
\end{Exercise}

\begin{Exercise}
Our ray tracer understands unions of objects, i.e. it is possible to create
a new shape by grouping two shapes together. There are other operations
like this, which yield more interesting results. Below you can see the result
of \emph{intersecting} two spheres to produce a lens.
\begin{center}
  \includegraphics[width=.5\linewidth]{csg.png}
\end{center}
What changes would you make to support intersection?
\end{Exercise}
